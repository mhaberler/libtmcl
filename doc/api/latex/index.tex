\hypertarget{index_NOTE}{}\section{NOTE}\label{index_NOTE}
\begin{DoxyVerb}
  This documentation is created with the doxygen source code documentation generator.
  It may be regenerated by calling "make doxygen-doc" in the main source tree if
  'doxgen' (http://www.stack.nl/~dimitri/doxygen/) is installed on the system.
 \end{DoxyVerb}
\hypertarget{index_intro_sec}{}\section{Introduction}\label{index_intro_sec}
The Trinamic Motion Control Language is a set of commands for the programming of Trinamic motor controller. For the direct control of a motor-\/controller board these commands have to be translated into a command number and bundled with the command arguments, the motor address and a checksum. The command set and the details of the programming process are documented in the \char`\"{}TMCL Reference
 Manual\char`\"{} from Trinamic, which can be downloaded from \href{http://www.trinamic.com/.}{\tt http://www.trinamic.com/.}

The aim of this library is to hide the low-\/level conversion and addressing issues from the user for easier programming. A typical program using libtmcl can be as simple as the following example (error checking omitted!).


\begin{DoxyCode}
 #include <tmcl/tmcl.h>

 int main(void) {

      TMCLInterface *SerialIface; // Stores the interface of the motor-controller
       board
      TMCLMotor *Motor;     // Stores information about the motor to be controlle
      d

      // Init interface structure
      tmcl_init_interface(&SerialIface, TMCL_RSXXX, NULL, NULL, NULL, NULL);

      // Open the interface
      tmcl_open_interface(SerialIface, "/dev/ttyS0");

      // Init motor structure
      tmcl_init_motor(&Motor, SerialIface, TMCM301, 1, 0, TMCL_RSXXX);

      // Rotate motor left
      tmcl_rol(TestMotor, 100);

      // Cleanup
      tmcl_deinit_motor(&TestMotor);
      tmcl_close_interface(TestIface);
      tmcl_deinit_interface(&TestIface);

 }
\end{DoxyCode}
\hypertarget{index_install_sec}{}\section{Installation}\label{index_install_sec}
There are no special prerequisites for 'libtmcl' installation. Normally it should be enough to call:


\begin{DoxyItemize}
\item ./configure
\item make
\end{DoxyItemize}

and than with 'root' privileges:


\begin{DoxyItemize}
\item make install
\end{DoxyItemize}

For details refer to the delivered 'INSTALL' file.\hypertarget{index_step1}{}\section{\char`\"{}Let the games begin!\char`\"{}}\label{index_step1}
\hypertarget{index_firststeps}{}\subsection{First steps}\label{index_firststeps}
The first things you need to know are:


\begin{DoxyItemize}
\item The model of your trinamic controller (see \hyperlink{tmcldefs_8h_a5b6ac18c2401b554e24fe3313eda6e9a}{TMCLModel} for supported models)
\item The module address and bank of you connected motor(s) (e.g. for the first motor of module \char`\"{}1\char`\"{}: address=1, bank=0)
\item The \hyperlink{tmcldefs_8h_a3c0af0cc3f62b9e4a1daea7839da918e}{interface type}. Currently only RS232/RS485 serial interfaces are supported. Custom communication functions (open, close, read, write) may be given to \hyperlink{interface_8h_a6234d4f85bda5c0132fdecde69565ef4}{tmcl\_\-init\_\-interface()} as pointers. See example01.c in the examples directory and have a look at \hyperlink{rsXXX_8c_source}{rsXXX.c} how to do this.
\end{DoxyItemize}

With these information first initialize and open you interface struct, e.g. for a serial RS232 connection at /dev/ttyS0:


\begin{DoxyCode}
 ...
 tmcl_init_interface(&SerialIface, TMCL_RSXXX, NULL, NULL, NULL, NULL);
 tmcl_open_interface(SerialIface, "/dev/ttyS0");
 ...
\end{DoxyCode}


Remember to check the return codes for errors! 'libtmcl' functions should return values $>$=0 for success and $<$0 for failure.

After that init your motor. In this case the motor is the first motor at a TMCM-\/301 module with address \char`\"{}1\char`\"{}:


\begin{DoxyCode}
 ...
 tmcl_init_motor(&Motor, SerialIface, TMCM301, 1, 0, TMCL_RSXXX);
 ...
\end{DoxyCode}


Again: Remember to check for errors!

Some commonly used functions are defined in \hyperlink{convenience_8h}{convenience.h}, which is included from \hyperlink{tmcl_8h}{tmcl.h} by default, e.g.


\begin{DoxyItemize}
\item Activate limit/reference switches: \hyperlink{convenience_8h_abfa4f22d0004e1c6a6edae3720cbcd11}{tmcl\_\-activate\_\-limit\_\-switch(TMCLMotor$\ast$, int limit\_\-switch)};
\item Doing a refsearch: \hyperlink{convenience_8h_a002fe6b01caeb3a1c4c245e15d96d5f0}{tmcl\_\-refsearch\_\-start(TMCLMotor$\ast$)} (Remember: Reference switches have to be active for that!)
\item Move to position X: \hyperlink{convenience_8h_ade9c1b3e4ada816b7bdd76d7fd2e1639}{tmcl\_\-move\_\-to\_\-pos\_\-abs(TMCLMotor$\ast$, int position)}
\item etc.
\end{DoxyItemize}\hypertarget{index_advanced}{}\section{\char`\"{}Advanced\char`\"{} usage}\label{index_advanced}
\hypertarget{index_sendcommand}{}\subsection{Send commands}\label{index_sendcommand}
There are more commands available than what are defined in \hyperlink{convenience_8h}{convenience.h} (see TMCL Reference for details). These functions can be accessed directly by the command number defined in the TMCL reference or, for greater readability, by a command define from \hyperlink{group__TMCLComm}{tmcldefs.h}

For example: If you want to submit the \char`\"{}Move to Position (relative)\char`\"{} command \char`\"{}by hand\char`\"{} you can can use the \hyperlink{motor_8h_a0994799e6eeee41f70093c081bdc7d0a}{tmcl\_\-send\_\-command}(...) function as follows (Again: No error checking is done here, but you should do it in real code!):


\begin{DoxyCode}
  ...
  TMCLCommand command;

  command.command = TMCL_MVP;     // TMCL_MVP is defined in tmcldefs.h as command
       number "4"
  command.type    = TMCL_MVP_REL; // Relative movement. Type is not necessary for
       all commands (see TMCL reference)
  command.value   = 100;          // Move 100 steps relative to current position

  tmcl_send_command(Motor, command, NULL);   // Submit the command. We do not exp
      ect a reply, so the last argument is NULL.
  ...
\end{DoxyCode}
\hypertarget{index_axispar}{}\subsection{Axis parameter}\label{index_axispar}
Axis parameters control the way the motor is moving, e.g. speed, numer of limit switches, etc. The parameters available can be seen in \hyperlink{tmcldefs_8h}{tmcldefs.h} or the TMCL reference.

For reading and writing of axis parameters the functions \hyperlink{motor_8h_acde6e9e540c95467c08ad479ca3627cd}{tmcl\_\-get\_\-axis\_\-parameter}(...) and \hyperlink{motor_8h_a55a0a1ad09c44b1386a528b4e74f3962}{tmcl\_\-set\_\-axis\_\-parameter}(...) exists.

Examples:


\begin{DoxyCode}
  ...
  int speed;

  // Get the current speed of the motor
  speed = tmcl_get_axis_parameter(motor, TMCL_AP_CURR_SPEED);

  // Set reference search speed. This is set as a fraction of the full positionin
      g speed,
  // e.g. 2 means: half the positioning speed, 4: quarter of the full positioning
       speed, etc.
  // See TMCL reference for details
  tmcl_set_axis_parameter(motor, TMCL_AP_RFS_SPEED, 2);

  ...
\end{DoxyCode}
 